\documentclass{article}
\begin{document}

Confounding is a distortion (inaccuracy) in the estimated measure of association that occurs when the primary exposure of interest is mixed up with some other factor that is associated with the outcome. Suppose that, for example, age is not observed when estimating the salary for men and women. In general, regardless of the sex, older people tend to make more money. Therefore, when age is not present one may make a wrong assumption about one gender making more money than the other. Also, living in a family with kids could make the career progression slower for a woman with 5 kids than 1.
\end{document}